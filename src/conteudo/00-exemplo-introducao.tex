%!TeX root=../tese.tex
%("dica" para o editor de texto: este arquivo é parte de um documento maior)
% para saber mais: https://tex.stackexchange.com/q/78101

%% ------------------------------------------------------------------------- %%

% "\chapter" cria um capítulo com número e o coloca no sumário; "\chapter*"
% cria um capítulo sem número e não o coloca no sumário. A introdução não
% deve ser numerada, mas deve aparecer no sumário. Por conta disso, este
% modelo define o comando "\chapter**".
\chapter**{Introdução}
\label{cap:introducao}

\enlargethispage{.5\baselineskip}

A centralização progressiva da internet tem sido um fenômeno notável ao longo das últimas décadas, a despeito da sua arquitetura que é fundamentalmente descentralizada. Tal consolidação, causada por diversos fatores econômicos, fez com que o controle da rede tenha se concentrado cada vez mais nas mãos de um pequeno conjunto de empresas muito grandes e poderosas. Tal consolidação levanta grandes preocupações sobre privacidade, segurança, e posse de dados pessoais, num cenário em que poucas aplicações amplamente utilizadas são de fato descentralizadas em arquitetura e governância.

Em uma análise mais detalhada, nota-se que esse fenômeno de centralização tem se apresentado em muitas componentes diferentes. Luciano Zembruzki junto de outros pesquisadores mostrou, por exemplo, como o mercado de hospedagem web se tornou extraordinariamente concentrado, e como apenas 10 provedores compunham quase toda a hospedagem dos domínios de topo de nível considerados em sua análise (Luciano Zembruzki et al., 2022). Por outro lado, Giovane Moura, junto de outros pesquisadores, notou que o Sistema de Nomes de Domínio

[VER A FONTE DO ARTIGO DE DNS AAAAAAA]