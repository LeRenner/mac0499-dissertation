%!TeX root=../tese.tex
%("dica" para o editor de texto: este arquivo é parte de um documento maior)
% para saber mais: https://tex.stackexchange.com/q/78101

%% ------------------------------------------------------------------------- %%

% "\chapter" cria um capítulo com número e o coloca no sumário; "\chapter*"
% cria um capítulo sem número e não o coloca no sumário. A introdução não
% deve ser numerada, mas deve aparecer no sumário. Por conta disso, este
% modelo define o comando "\chapter**".
\chapter**{Introdução}
\label{cap:introducao}

\enlargethispage{.5\baselineskip}

A centralização progressiva da internet tem sido um fenômeno notável ao longo das últimas décadas, a despeito da sua arquitetura que é fundamentalmente descentralizada. Tal consolidação, causada por diversos fatores econômicos e sociais, fez com que o controle da rede tenha se concentrado cada vez mais nas mãos de um pequeno conjunto de empresas extremamente influentes. Tal situação levanta grandes preocupações sobre privacidade, segurança, e posse de dados pessoais, num cenário em que poucas aplicações amplamente utilizadas são de fato descentralizadas em arquitetura e governância.

Em uma análise mais detalhada, nota-se que esse fenômeno de centralização tem se apresentado em muitas componentes diferentes. Luciano Zembruzki junto de outros pesquisadores mostrou, por exemplo, como o mercado de hospedagem web se tornou extraordinariamente concentrado, e como apenas 10 provedores compunham quase toda a hospedagem dos domínios de topo de nível considerados em sua análise (Luciano Zembruzki et al., 2022). Por outro lado, Giovane Moura, Sebastian Castro, e outros em um estudo notaram que o Sistema de Nomes de Domínio tem sido modificado de maneira similar, apresentando "concentração notável" (Giovane Moura et al., 2020). Outros artigos relatam comportamentos similares em Content Delivery Networks (CDNs) e na propagação da versão mais recente do protocolo de segurança Transport Layer Security (TLS) 1.3 (Kevin Vermeulen et al., 2023) (Ralph Holz et al., 2020). Assim, percebe-se que esse processo é muito difundido, e afeta multiplas facetas técnicas da internet moderna.

Por outro lado, mesmo numa perspectiva social e econômica a rede mundial de computadores tem passado por um constande processo de consolidação corporativa, fusões entre empresas, e uma presença cada vez maior de tecnologias proprietárias e ecossistemas fechados que evitam a migração dos usuários para serviços concorrentes. Tal processo é descrito em excelente detalhe por Ulrich Dolata no quinto capítulo do livro "Collectivity and Power on the Internet: A Social Perspective" (Ulrich Dolata, 2018), relatando como empresas buscam intencionalmente evitar concorrência e livre mercado, empregando as técnicas anteriormente descritas. 

Nesse sentido, a presença cada vez menor de protocolos de compartilhamento de arquivos e mensagens descentralizadas é um processo pelo menos parcialmente influenciado por entidades comerciais, uma vez que companhias tem diversas vantagens e incentivos para controlar e manejar cada vez mais dados de seus clientes, ao invés de delegá-los a um sistema autônomo. Além disso, sistemas abertos e controlados por seus usuários muitas vezes requerem um nível mais elevado de conhecimento técnico para serem utilizados, tornando-se menos preferíveis para o consumidor médio. É muito comum que soluções comerciais tenham mais suporte e sejam mais convenientes e simples de se utilizar, adquirindo assim mais clientes ao longo prazo. Assim, um sistema distribuido que consiga lidar com todos os aspectos técnicos da transmissão de mensagens de forma descentralizada pode ser uma solução preferível para um usuário que prefere não ter seus dados pessoais controlados por uma entidade com fins lucrativos. 

Nexte contexto, esta monografia tem como objetivo uma análise técnica das limitações de sistemas abertos de envio de mensagens, considerando aspectos como a arquitetura descentralizada, sincronicidade, confiabilidade na entrega de mensagens e resistência a atores mal-intencionados. Embora existam diversas implementações de protocolos descentralizados de código aberto, este trabalho não tem como objetivo criar um protocolo completamente abrangente que substitua todos os demais. Em vez disso, ele busca uma análise crítica e abrangente dessas soluções, destacando suas vantagens e limitações. Essa avaliação é fundamental para informar futuros estudos e o desenvolvimento de novos protocolos. Além disso, a criação de uma prova de conceito desempenha um papel essencial neste estudo, ao permitir a aplicação prática das ideias discutidas, demonstrando concretamente seus méritos e desafios.
