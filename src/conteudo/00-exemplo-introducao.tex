%!TeX root=../tese.tex
%("dica" para o editor de texto: este arquivo é parte de um documento maior)
% para saber mais: https://tex.stackexchange.com/q/78101

%% ------------------------------------------------------------------------- %%

% "\chapter" cria um capítulo com número e o coloca no sumário; "\chapter*"
% cria um capítulo sem número e não o coloca no sumário. A introdução não
% deve ser numerada, mas deve aparecer no sumário. Por conta disso, este
% modelo define o comando "\chapter**".
\chapter**{Introdução}
\label{cap:introducao}

\enlargethispage{.5\baselineskip}

A centralização progressiva da internet tem sido um fenômeno notável ao longo das últimas décadas, a despeito da sua arquitetura que é fundamentalmente descentralizada. Tal consolidação, causada por diversos fatores econômicos e sociais, fez com que o controle da rede tenha se concentrado cada vez mais nas mãos de um pequeno conjunto de empresas extremamente influentes. Tal situação levanta grandes preocupações sobre privacidade, segurança, e posse de dados pessoais, num cenário em que poucas aplicações amplamente utilizadas são de fato descentralizadas em arquitetura e governância.

Em uma análise mais detalhada, nota-se que esse fenômeno de centralização tem se apresentado em muitas componentes diferentes. Luciano Zembruzki junto de outros pesquisadores mostrou, por exemplo, como o mercado de hospedagem web se tornou extraordinariamente concentrado, e como apenas 10 provedores compunham quase toda a hospedagem dos domínios de topo de nível considerados em sua análise (Luciano Zembruzki et al., 2022). Por outro lado, Giovane Moura, Sebastian Castro, e outros em um estudo notaram que o Sistema de Nomes de Domínio tem sido modificado de maneira similar, apresentando "concentração notável" (Giovane Moura et al., 2020). Outros artigos relatam comportamentos similares em Content Delivery Networks (CDNs) e na propagação da versão mais recente do protocolo de segurança Transport Layer Security (TLS) 1.3 (Kevin Vermeulen et al., 2023) (Ralph Holz et al., 2020). Assim, percebe-se que esse processo é muito difundido, e afeta multiplas facetas técnicas da internet moderna.

Por outro lado, mesmo numa perspectiva social e econômica a rede mundial de computadores tem passado por um constande processo de consolidação corporativa, fusões entre empresas, e uma presença cada vez maior de tecnologias proprietárias e ecossistemas fechados que evitam a migração dos usuários para serviços concorrentes. Tal processo é descrito em excelente detalhe por Ulrich Dolata no quinto capítulo do livro "Collectivity and Power on the Internet: A Social Perspective", relatando como empresas buscam intencionalmente evitar concorrência e livre mercado, empregando as técnicas anteriormente descritas. 

Nesse sentido, a presença cada vez menor de protocolos de compartilhamento de arquivos e mensagens descentralizadas é um processo pelo menos parcialmente influenciado por entidades comerciais, uma vez que companhias tem diversas vantagens e incentivos para controlar e manejar cada vez mais dados de seus clientes, ao invés de delegá-los a um sistema autônomo. Por outro lado, sistemas abertos e controlados por seus usuários tem desvantagens e inconveniências associados a eles que o usuário comum muitas vezes pode preferir evitar. Soluções comerciais vão ter mais suporte, ser mais convenientes e simples de se utilizar, e requerir menos conhecimento técnico para serem utilizados, adquirindo assim mais clientes ao longo prazo. Assim, um sistema distribuido que consiga lidar com todos os aspectos técnicos da transmissão de mensagens de forma descentralizada pode ser uma solução preferível para pessoas que não tem conhecimento técnico muito avançado, mas que preferem não ter seus dados pessoais controlados por uma entidade com fins lucrativos. 

Nesse contexto, este trabalho tem como objetivo o desenvolvimento de um sistema descentralizado de compartilhamento de arquivos que tenha criptografia de ponta-a-ponta e que seja simples de se usar, mesmo para um usuário com poucos conhecimentos sobre computadores. Esta monografia será uma exploração dos desafios técnicos associados a este desenvolvimento, e considerações sobre tal sistema, relatando os motivos por trás da arquitetura escolhida. 