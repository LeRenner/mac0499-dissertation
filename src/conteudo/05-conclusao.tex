\chapter{Conclusão e Trabalhos Futuros}

O protótipo que foi desenvolvido mostra como um protocolo híbrido de comunicação, utilizando tanto a rede do \textit{Tor} quanto conexões diretas entre usuários, pode permitir a comunicação entre usuários sem a atuação de nenhuma entidade intermediária controlando o fluxo de mensagens. Todavia, essa implementação é limitada por problemas de latência e estabilidade de conexão, que são inerentes ao uso do \textit{Tor}. Além disso, enquanto este protocolo não depende de voluntários dedicados a este específico protocolo, ele faz uso de uma infraestrutura de rede que é mantida por voluntários. Modificações no protocolo, mudanças na legislação, ou até mesmo uma diminuição no uso e popularidade do \textit{Tor} podem tornar este protocolo inutilizável.

Além disso, a implementação de \textit{UPnP} utilizada não é ideal e pode ser considerada mais próxima de um improviso do que uma implementação de produção. A biblioteca \textit{miniupnpc}, que foi utilizada inicialmente, apresentou problemas de \textit{timeout} e de inconsistência nas respostas dos roteadores. A implementação atual, que utiliza o executável \textit{upnpc}, é mais estável, mas ainda apresenta problemas de inconsistência nas respostas dos roteadores. Enquanto o \textit{UPnP} é um protocolo muito interessante e com muito potencial em implementações de protocolos descentralizados e \textit{P2P}, é necessário que este protocolo tenha mais adoção e que as implementações sejam mais estáveis e consistentes.

A implementação de um protocolo de comunicação \textit{P2P} é um desafio técnico, e a implementação de um protocolo de comunicação \textit{P2P} que seja seguro e anônimo é ainda mais desafiadora. Este protótipo é uma prova de conceito que mostra que é possível implementar um protocolo de comunicação \textit{P2P} que seja seguro e anônimo, mas que ainda tem muitas limitações práticas.

\section{Trabalhos Futuros}

O protótipo implementado não abrange completamente todas as configurações possíveis de rede. Um excelente exemplo que não foi abrangido nesta implementação, por exemplo, é um computador que tenha um endereço de IP público em sua interface de rede. Neste caso, o programa não precisaria de \textit{UPnP} para abrir portas no roteador e poderia se comunicar diretamente com outros usuários. Além disso, a implementação atual não permite a comunicação direta entre usuários que estão atrás de \textit{NATs} simétricos, que são \textit{NATs} que não permitem a comunicação entre dois usuários que estão atrás de \textit{NATs} diferentes. Uma implementação futura poderia utilizar técnicas diferentes, como \textit{hole punching}, para permitir a comunicação entre usuários que estão atrás de \textit{NATs} simétricos.

Mesmo que a maioria dos programas de envio de mensagens popularmente usados sejam centralizados, como o \textit{WhatsApp}, o \textit{Telegram} e o \textit{Facebook Messenger}, a exploração de protocolos descentralizados e \textit{P2P} é extremamente interessante em um cenário de cada vez maior centralização de poder na \textit{internet}. Grandes plataformas \textit{online} já tiveram momentos de instabilidade, e situações como essas permitem que plataformas descentralizadas sejam mais atrativas para usuários que buscam controle e privacidade de seus dados pessoais.