%!TeX root=../tese.tex
%("dica" para o editor de texto: este arquivo é parte de um documento maior)
% para saber mais: https://tex.stackexchange.com/q/78101

% As palavras-chave são obrigatórias, em português e em inglês, e devem ser
% definidas antes do resumo/abstract. Acrescente quantas forem necessárias.
\palavraschave{Sistemas descentralizados. Protocolos de mensagens. Criptografia. Resiliência a falhas}

\keywords{Descentralized systems. Messaging protocols. Cryptography. Fault tolerance}

% O resumo é obrigatório, em português e inglês. Estes comandos também
% geram automaticamente a referência para o próprio documento, conforme
% as normas sugeridas da USP.
\resumo{
    Este trabalho realiza uma análise de protocolos descentralizados de envio de mensagens modernos, classificando-os de acordo com suas arquiteturas. Em seguida, é proposta a implementação de um protótipo funcional que combina as melhores características desses protocolos. O protótipo desenvolvido é compatível com configurações de rede comuns, é resistente a ataques de voluntários mal-intencionados e permite que pessoas se encontrem na rede utilizando apenas seus nomes de usuário.
}

\abstract{
    This work presents an analysis of modern decentralized messaging protocols, classifying them according to their architectures. Following this, a functional prototype is proposed that combines the best features of these protocols. The developed prototype is compatible with common network configurations, resilient to attacks by malicious volunteers, and allows users to find each other on the network using only their usernames.
}
