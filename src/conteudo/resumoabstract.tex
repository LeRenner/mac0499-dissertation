%!TeX root=../tese.tex
%("dica" para o editor de texto: este arquivo é parte de um documento maior)
% para saber mais: https://tex.stackexchange.com/q/78101

% As palavras-chave são obrigatórias, em português e em inglês, e devem ser
% definidas antes do resumo/abstract. Acrescente quantas forem necessárias.
\palavraschave{Sistemas descentralizados. Protocolos de mensagens. Criptografia. Resiliência a falhas}

\keywords{Descentralized systems. Messaging protocols. Cryptography. Fault tolerance}

% O resumo é obrigatório, em português e inglês. Estes comandos também
% geram automaticamente a referência para o próprio documento, conforme
% as normas sugeridas da USP.
\resumo{
Essa monografia descreve, em detalhe, os desafios na implementação de um protocolo
de envio de mensagens que garanta criptografia de ponta a ponta, descentralização,
assincronia, e resiliência a falhas na rede causadas por atores mal intencionados. 
Uma vez que muitas dessas características são fundamentalmente incompatíveis entre si,
é buscado encontrar os melhores compromissos entre elas, entender suas vantagens e
limitações, e eventualmente implementar um protótipo funcional que ponha em prática
as ideias discutidas.
}

\abstract{
This article describes, in detail, the challenges in implementing a messaging protocol
that guarantees end-to-end encryption, decentralization, asynchrony, and fault tolerance
in the network caused by malicious actors. Since many of these characteristics are
fundamentally incompatible with each other, the goal is to find the best compromises
between them, understand their advantages and limitations, and eventually implement a
functional prototype that puts into practice the ideas discussed.
}
