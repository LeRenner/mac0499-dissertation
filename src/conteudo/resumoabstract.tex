%!TeX root=../tese.tex
%("dica" para o editor de texto: este arquivo é parte de um documento maior)
% para saber mais: https://tex.stackexchange.com/q/78101

% As palavras-chave são obrigatórias, em português e em inglês, e devem ser
% definidas antes do resumo/abstract. Acrescente quantas forem necessárias.
\palavraschave{Sistemas descentralizados. Protocolos de mensagens. Criptografia. Resiliência a falhas}

\keywords{Descentralized systems. Messaging protocols. Cryptography. Fault tolerance}

% O resumo é obrigatório, em português e inglês. Estes comandos também
% geram automaticamente a referência para o próprio documento, conforme
% as normas sugeridas da USP.
\resumo{
    Sistemas distribuídos são parte essencial da Internet, possibilitando a troca de mensagens, arquivos, ou até mesmo dinheiro entre pessoas ao redor do mundo. Tais sistemas podem ser desenhados de forma centralizada ou descentralizada, a depender da existência de uma entidade central que coordene as interações entre os participantes. Enquanto sistemas descentralizados não possuem ponto central de falha, eles possuem uma série de dificuldades e desafios técnicos para sua implementação. Este trabalho realiza uma análise de protocolos descentralizados de envio de mensagens modernos, classificando-os de acordo com suas arquiteturas. Em seguida, é proposta a implementação de um protótipo funcional que combina as melhores características desses protocolos para permitir a troca de mensagens entre usuários na Internet. O protótipo desenvolvido é compatível com configurações de rede comuns, permite que os usuários utilizem nomes de usuário simples, e alterna entre diferentes formas de conexão dependendo das condições de rede que os usuários se encontram. Além disso, técnicas de criptografia e de roteamento por caminhos aleatórios são afixados para garantir a privcidade e confidencialidade das mensagens trocadas.
}

\abstract{
    Distributed systems are an essential part of the Internet, enabling the exchange of messages, files, and even money between individuals worldwide. These systems can be designed in either a centralized or decentralized manner, depending on the presence of a central entity that coordinates interactions among participants. While decentralized systems lack a single point of failure, they present numerous technical difficulties and implementation challenges. This work conducts an analysis of modern decentralized messaging protocols, classifying them based on their architectures. Subsequently, the implementation of a functional prototype is proposed, combining the best features of these protocols to facilitate message exchange between Internet users. The developed prototype is compatible with common network configurations, allows users to employ simple usernames, and dynamically switches between different connection methods depending on network conditions. Additionally, encryption techniques and random path routing are employed to ensure the privacy and confidentiality of exchanged messages.
}
