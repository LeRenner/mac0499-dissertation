%!TeX root=../tese.tex
%("dica" para o editor de texto: este arquivo é parte de um documento maior)
% para saber mais: https://tex.stackexchange.com/q/78101

% As palavras-chave são obrigatórias, em português e em inglês, e devem ser
% definidas antes do resumo/abstract. Acrescente quantas forem necessárias.
\palavraschave{Sistemas descentralizados. Protocolos de mensagens. Criptografia. Resiliência a falhas}

\keywords{Descentralized systems. Messaging protocols. Cryptography. Fault tolerance}

% O resumo é obrigatório, em português e inglês. Estes comandos também
% geram automaticamente a referência para o próprio documento, conforme
% as normas sugeridas da USP.
\resumo{
    Sistemas distribuidos são parte essencial da Internet moderna, possibilitando a troca de mensagens, arquivos, ou até mesmo dinheiro entre pessoas ao redor do mundo. Tais sistemas podem ser desenhados de forma centralizada ou descentralizada, a depender da existência de uma entidade central que coordene as interações entre os participantes. Enquanto sistemas distribuidos não possuem ponto central de falha, eles possuem uma série de dificuldades e desafios técnicos para sua implementação. Este trabalho realiza uma análise de protocolos descentralizados de envio de mensagens modernos, classificando-os de acordo com suas arquiteturas. Em seguida, é proposta a implementação de um protótipo funcional que combina as melhores características desses protocolos. O protótipo desenvolvido é compatível com configurações de rede comuns, permite que os usuários utilizem nomes de usuário simples, e alterna entre diferentes formas de conexão dependendo das condições de rede que os usuários se encontram.
}

\abstract{
    Distributed systems are an essential part of the modern Internet, enabling the exchange of messages, files, or even money between people around the world. Such systems can be designed in a centralized or decentralized manner, depending on the existence of a central entity that coordinates interactions between participants. While distributed systems do not have a single point of failure, they have a series of difficulties and technical challenges for their implementation. This work presents an analysis of modern decentralized messaging protocols, classifying them according to their architectures. Following this, a functional prototype is proposed that combines the best features of these protocols. The developed prototype is compatible with common network configurations, resilient to attacks by malicious volunteers, and allows users to find each other on the network using only their usernames.
}
