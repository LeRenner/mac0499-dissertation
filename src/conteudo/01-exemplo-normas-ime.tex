%!TeX root=../tese.tex
%("dica" para o editor de texto: este arquivo é parte de um documento maior)
% para saber mais: https://tex.stackexchange.com/q/78101

\chapter{Implementações já existentes}

No desenvolvimento de um protocolo de mensagens descentralizado, o primeiro passo essencial é analisar as soluções de código aberto já existentes. Essas implementações fornecem uma base para compreender os desafios práticos, as abordagens de arquitetura e as estratégias de mitigação de problemas que foram desenvolvidas e testadas pela comunidade. Os protocolos analisados foram selecionados por serem abertos, descentralizados, e por terem uma arquitetura de rede singular, sendo relevante para este estudo. Os protocolos analisados foram categorizados pela sua arquitetura de rede, e divididos em três grupos principais: federados, peer-to-peer com relays, e peer-to-peer sem relays. 

Os protocolos analisados federados são:
\begin{itemize}
  \item \href{https://matrix.org/}{Matrix}
  \item \href{https://xmpp.org/}{XMPP}
  \item \href{https://en.wikipedia.org/wiki/Email}{E-Mail}
\end{itemize}

Os protocolos analisados peer-to-peer com relays que não usam Tor são:
\begin{itemize}
  \item \href{https://threema.ch/}{Threema}
  \item \href{https://simplex.chat/}{SimpleX}
  \item \href{https://www.irc.org/}{IRC}
  \item \href{https://lora-alliance.org/}{LoRaWan}
\end{itemize}

Os protocolos analisados peer-to-peer com relays que usam Tor são:
\begin{itemize}
  \item \href{https://cwtch.im/}{Cwtch}
  \item \href{https://ricochet.im/}{Ricochet}
  \item \href{https://briarproject.org/}{Briar}
\end{itemize}

Os protocolos analisados peer-to-peer sem relays são:
\begin{itemize}
  \item \href{https://scuttlebutt.nz/}{Secure Scuttlebutt}
  \item \href{https://tox.chat/}{Tox}
  \item \href{https://bitmessage.org/}{Bitmessage}
\end{itemize}

A análise de cada protocolo é focada na arquitetura de rede, e não considera em detalhes outras funcionalidades como chamadas de voz, videoconferência, ou compartilhamento de arquivos. A seguir, são apresentadas as análises dos protocolos selecionados.

\section{Protocolos Federados}

Na arquitetura federada de envio de mensagens, os usuários escolhem servidores específicos para armazenar e transmitir suas mensagens. Esses servidores se comunicam entre si para rotear as mensagens destinadas a usuários em outros servidores. Como o servidor é responsável pela entrega das mensagens, os protocolos federados permitem o envio assíncrono de mensagens. Em geral, os endereços dos usuários nesses sistemas consistem em um nome de usuário combinado com o endereço do servidor.

\subsection{Matrix}

O protocolo Matrix é um sistema de comunicação descentralizado e federado projetado para suportar mensagens instantâneas e colaboração em tempo real. Desenvolvido para ser uma solução aberta e interoperável, o Matrix visa fornecer uma infraestrutura robusta para comunicação em diversos contextos, desde chat e chamadas de voz até a troca de arquivos.

\begin{itemize}
  \item \textbf{Tipo de endereço}: @usuário:servidor
  \item \textbf{Suporte a criptografia de ponta a ponta}: Sim
\end{itemize}

\subsection{XMPP}

O protocolo XMPP (Extensible Messaging and Presence Protocol) é um protocolo federado de mensagens instantâneas e presença. Desenvolvido originalmente pela Jabber Inc., o XMPP é um protocolo aberto e extensível que permite a comunicação entre diferentes servidores e clientes.

\begin{itemize}
  \item \textbf{Tipo de endereço}: usuário@servidor
  \item \textbf{Suporte a criptografia de ponta a ponta}: Sim
\end{itemize}

\subsection{E-Mail}

O E-Mail é um protocolo de comunicação assíncrona que permite o envio e recebimento de mensagens de texto, anexos e outros arquivos.

\begin{itemize}
  \item \textbf{Tipo de endereço}: usuário@servidor
  \item \textbf{Suporte a criptografia de ponta a ponta}: Sim
\end{itemize}
