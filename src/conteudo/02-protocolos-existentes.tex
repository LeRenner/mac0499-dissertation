%!TeX root=../tese.tex
%("dica" para o editor de texto: este arquivo é parte de um documento maior)
% para saber mais: https://tex.stackexchange.com/q/78101

\chapter{Implementações já existentes}

Este capítulo da monografia realizará a análise de implementações já existentes de protocolos descentralizados de envio de mensagens pela comunidade de código aberto. Essas implementações fornecem uma base para compreender os desafios práticos, as abordagens de arquitetura e as estratégias de mitigação de problemas que foram desenvolvidas e testadas pela comunidade. Os protocolos analisados foram selecionados por serem abertos, descentralizados, e por terem uma arquitetura de rede singular, sendo relevante para este estudo. Os protocolos analisados foram categorizados pela sua arquitetura de rede, e divididos em quatro grupos principais: federados, peer-to-peer com relays que usam tor, peer-to-peer com relays que não usam tor, e peer-to-peer sem relays. 

Relays são servidores intermediários que encaminham mensagens entre usuários, e são utilizados para contornar restrições de rede, como NATs e firewalls. A rede Tor é uma rede de anonimização que permite a comunicação anônima e segura entre usuários, mesmo quando eles estão em redes censuradas, com NATs, ou com firewalls restritivos.

Os protocolos analisados federados são:
\begin{itemize}
  \item \href{https://matrix.org/}{Matrix}
  \item \href{https://xmpp.org/}{XMPP}
  \item \href{https://en.wikipedia.org/wiki/Email}{E-Mail}
\end{itemize}

Os protocolos analisados peer-to-peer com relays que não usam Tor são:
\begin{itemize}
  \item \href{https://simplex.chat/}{SimpleX}
  \item \href{https://www.irc.org/}{IRC}
  \item \href{https://lora-alliance.org/}{LoRaWan}
\end{itemize}

Os protocolos analisados peer-to-peer com relays que usam Tor são:
\begin{itemize}
  \item \href{https://cwtch.im/}{Cwtch}
  \item \href{https://ricochet.im/}{Ricochet}
  \item \href{https://briarproject.org/}{Briar}
\end{itemize}

Os protocolos analisados peer-to-peer sem relays são:
\begin{itemize}
  \item \href{https://scuttlebutt.nz/}{Secure Scuttlebutt}
  \item \href{https://tox.chat/}{Tox}
  \item \href{https://bitmessage.org/}{Bitmessage}
\end{itemize}

A análise de cada protocolo é focada na arquitetura de rede, e não considera em detalhes outras funcionalidades como chamadas de voz, videoconferência, ou compartilhamento de arquivos. A seguir, são apresentadas as análises dos protocolos selecionados.

\section{Protocolos Federados}

Na arquitetura federada de envio de mensagens, os usuários escolhem servidores específicos para armazenar e transmitir suas mensagens. Esses servidores se comunicam entre si para rotear as mensagens destinadas a usuários em outros servidores. Como o servidor é responsável pela entrega das mensagens, os protocolos federados permitem o envio assíncrono de mensagens. Em geral, os endereços dos usuários nesses sistemas consistem em um nome de usuário combinado com o endereço do servidor.

\subsection{Matrix}

O protocolo Matrix é um sistema de comunicação descentralizado e federado projetado para suportar mensagens instantâneas e colaboração em tempo real. Desenvolvido para ser uma solução aberta e interoperável, o Matrix visa fornecer uma infraestrutura robusta para comunicação em diversos contextos, desde chat e chamadas de voz até a troca de arquivos \cite{matrixspec}.

\begin{itemize}
  \item \textbf{Tipo de endereço}: @usuário:servidor
  \item \textbf{Suporte a criptografia de ponta a ponta}: Sim
\end{itemize}

\subsection{XMPP}

O protocolo XMPP (Extensible Messaging and Presence Protocol) é um protocolo federado de mensagens instantâneas e presença. Desenvolvido originalmente pela Jabber Inc., o XMPP é um protocolo aberto e extensível que permite a comunicação entre diferentes servidores e clientes \cite{xmppspec}.

\begin{itemize}
  \item \textbf{Tipo de endereço}: usuário@servidor
  \item \textbf{Suporte a criptografia de ponta a ponta}: Sim
\end{itemize}

\subsection{E-Mail}

O E-Mail é um protocolo de comunicação assíncrona que permite o envio e recebimento de mensagens de texto, anexos e outros arquivos \cite{rfc5321}.

\begin{itemize}
  \item \textbf{Tipo de endereço}: usuário@servidor
  \item \textbf{Suporte a criptografia de ponta a ponta}: Sim
\end{itemize}

\section{Protocolos Peer-to-Peer com Relays}

Na arquitetura peer-to-peer com relays, os usuários se conectam diretamente uns aos outros, e armazenam localmente suas mensagens. Para contornar limitações de rede, como camadas de Network Adress Translation (NAT) e firewalls, os usuários podem usar relays para intermediar a comunicação. Os relays são apenas responsáveis por encaminhar diretamente as mensagens, as apagando quando elas são entregues. Em alguns protocolos, esses relays podem armazenar a mensagem por um curto período de tempo, até o destinatário estar disponível para recebê-la.

Alguns desses protocolos utilizam como relay a rede Tor, que é uma rede de anonimização que permite a comunicação anônima e segura entre usuários, mesmo quando eles encontram-se em redes censuradas, com NATs, ou com firewalls restritivos.

\subsection{Protocolos Peer-to-Peer com Relays que não usam Tor}

\subsubsection{SimpleX}

O SimpleX é um protocolo de mensagens instantâneas peer-to-peer que utiliza relays para intermediar a comunicação entre usuários. Sua característica distinta é a sua completa ausência de um identificador de usuário, de modo que usuários podem iniciar uma conversa utilizando um simples código de convite \cite{simplex}.

\begin{itemize}
  \item \textbf{Tipo de endereço}: Código de convite
  \item \textbf{Suporte a criptografia de ponta a ponta}: Sim
  \item \textbf{Assíncrono}: Sim
\end{itemize}

\subsubsection{IRC}

O IRC (Internet Relay Chat) é um protocolo de comunicação em tempo real que permite a troca de mensagens em canais de chat específicos. As mensagens só podem ser recebidas de forma síncrona, uma vez que o servidor se comporta como um broadcast, enviando as mensagens para todos os usuários conectados \cite{rfc2810}.

\begin{itemize}
  \item \textbf{Tipo de endereço}: endereço de servidor
  \item \textbf{Suporte a criptografia de ponta a ponta}: A maioria dos clientes não suporta
  \item \textbf{Assíncrono}: Não
\end{itemize}

\subsubsection{LoRaWan}

O LoRaWan é um protocolo de comunicação de longo alcance e baixa potência que permite a comunicação entre dispositivos de Internet of Things (IoT). O protocolo se utiliza tanto de LoRa, que é uma tecnologia de modulação de rádio, quanto de redes de relays para transmitir mensagens entre dispositivos pela internet \cite{lorawan}.

\begin{itemize}
  \item \textbf{Tipo de endereço}: endereço de dispositivo
  \item \textbf{Suporte a criptografia de ponta a ponta}: Sim
  \item \textbf{Assíncrono}: Sim
\end{itemize}

\subsection{Protocolos Peer-to-Peer com Relays que usam Tor}

\subsubsection{Cwtch}

O Cwtch é um protocolo de mensagens instantâneas peer-to-peer que utiliza a rede Tor para intermediar a comunicação entre usuários. Cada usuário roda um serviço oculto no tor, e os usuários se conectam diretamente uns aos outros através da rede Tor \cite{cwtch}.

\begin{itemize}
  \item \textbf{Tipo de endereço}: Endereço do serviço oculto
  \item \textbf{Suporte a criptografia de ponta a ponta}: Sim
  \item \textbf{Assíncrono}: Não
\end{itemize}

\subsubsection{Ricochet}

O Ricochet funciona de maneira muito similar ao Cwtch. Ele utiliza a rede Tor para intermediar a comunicação entre usuários, e não requer um servidor central para funcionar \cite{ricochet}.

\begin{itemize}
  \item \textbf{Tipo de endereço}: Endereço do serviço oculto
  \item \textbf{Suporte a criptografia de ponta a ponta}: Sim
  \item \textbf{Assíncrono}: Não
\end{itemize}

\subsubsection{Briar}

O Briar é um protocolo muito versátil, e que permite o envio de mensagens através de Bluetooth, Wi-Fi, ou da internet. Quando as mensagens são enviadas pela internet, ele utiliza a rede Tor para intermediar a comunicação entre usuários, e é capaz de funcionar mesmo em redes censuradas \cite{briar}.

\begin{itemize}
  \item \textbf{Tipo de endereço}: Chave pública e privada (por meio de um QR code)
  \item \textbf{Suporte a criptografia de ponta a ponta}: Sim
  \item \textbf{Assíncrono}: Sim (com um dispositivo secundário rodando o serviço Briar Mailbox)
\end{itemize}

\section{Protocolos Peer-to-Peer sem Relays}

Na arquitetura peer-to-peer sem relays, os usuários se conectam diretamente, ou através de outros usuários, mas o protocolo não tem computadores dedicados para o intermédio de mensagens. Em geral, esses protocolos precisam que o usuário inicialmente saiba o endereço de outros usuários para poder se comunicar com eles, e estes protocolos são mais vulneráveis a restrições na rede, como NATs e firewalls.

\subsection{Secure Scuttlebutt}

O Secure Scuttlebutt é um protocolo de rede social descentralizada, que também permite o envio de mensagens privadas entre usuários. Nele, os usuários armazenam as próprias mensagens, e as replicam para outros usuários que estão conectados a eles. Geralmente, usuários vão receber um endereço de um amigo para se conectar, e a partir daí vão receber mensagens de amigos de amigos, e assim por diante \cite{scuttlebutt}.

\begin{itemize}
  \item \textbf{Tipo de endereço}: endereço, porta e chave pública
  \item \textbf{Suporte a criptografia de ponta a ponta}: Sim
  \item \textbf{Assíncrono}: Sim, por meio de replicação de mensagens
\end{itemize}

\subsection{Tox}

O tox é um protocolo de mensagens que primariamente não utiliza servidores intermediários, mas que pode recorrer a relays quando ambos os usuários não podem receber conexões externas. Ele permite o envio de mensagens, chamadas de voz e videoconferência \cite{toxcore}.

\begin{itemize}
  \item \textbf{Tipo de endereço}: Chave pública
  \item \textbf{Suporte a criptografia de ponta a ponta}: Sim
  \item \textbf{Assíncrono}: Não
\end{itemize}

\subsection{Bitmessage}

Bitmessage é um protocolo que apresenta similaridades com o Bitcoin, uma vez que usuários precisam resolver um desafio de prova de trabalho para enviar mensagens. Nele, mensagens percorrem a rede inteira até chegar ao destinatário \cite{bitmessage}.

\begin{itemize}
  \item \textbf{Tipo de endereço}: Chave pública
  \item \textbf{Suporte a criptografia de ponta a ponta}: Sim
  \item \textbf{Assíncrono}: Sim
\end{itemize}

