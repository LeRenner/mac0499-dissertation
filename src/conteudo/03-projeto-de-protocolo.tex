%!TeX root=../tese.tex
%("dica" para o editor de texto: este arquivo é parte de um documento maior)
% para saber mais: https://tex.stackexchange.com/q/78101

\chapter{Projeto de Protocolo}

Agora que já definimos conceitos importantes e analisamos as principais implementações de protocolos de mensagens descentralizados, podemos definir um projeto de protocolo que possua parte das características anteriormente definidas. Este capítulo terá como objetivo explicar as decisões tomadas durante o projeto, e quais compromissos foram tomados para atender a esses requisitos.

\section{Objetivos para o projeto}

Como não é possível cumprir todos os requisitos possíveis ao desenvolver um protocolo descentralizado, vamos precisar encontrar um bom compromisso de funcionalidades ao desenhar sua arquitetura. Para este projeto vamos designar que as seguintes características são as mais importantes:

\begin{itemize}
    \item \textbf{Facilidade na configuração}: O protocolo não deve precisar de configurações avançadas de rede, e deve funcionar em qualquer conexão residencial comum.
    \item \textbf{Pouca dependência em voluntários}: O protocolo deve ser razoavelmente resistente a maus atores, e deve evitar depender de voluntários para funcionar.
    \item \textbf{Endereço de usuários estável}: O protocolo deve permitir que usuários tenham um endereço único e estável, para que possam ser identificados de maneira fácil.
    \item \textbf{Privacidade}: O protocolo deve evitar expor informações pessoais dos usuários, e deve criptografar os dados do usuário quando possível.
\end{itemize}

Considerando esses requisitos, podemos já definir algumas características da nossa prova de conceito. 

\section{Arquitetura do protocolo}

Como discutido no primeiro capítulo, as arquiteturas dos protocolos de rede analisados podem ser divididas em quatro grupos diferentes: federados, indiretos por meio de relays específicos da rede, indiretos por meio da rede do Tor, e peer-to-peer. Podemos eliminar algumas dessas possibilidades com base nos requisitos que definimos.

Como consideramos que facilidade na configuração é um requisito importante, podemos concluir que não devemos desenvolver um serviço de arquitetura federada. Esse tipo de arquitetura depende de servidores mais centralizados, ou que os usuários tenham conhecimento técnico e uma configuração de rede muito específica para que possam hospedar seu próprio servidor. Como buscamos um protocolo que consiga ser configurado sem esforço e que funcione em qualquer configuração de rede, vamos considerar outra arquitetura para esse projeto.

Queremos evitar que o protocolo seja muito dependente de relays específicos da rede para funcionar. Esses relays dependem de um serviço centralizado para que o usuário os encontre, e eles podem ser fácilmente controlados por maus atores especialmente quando o protocolo não possui muitos usuários. Como queremos que o protocolo não possua essas vulnerabilidades, não vamos desenhar o protocolo com essa arquitetura.

O protocolo do Tor é uma opção interessante, pois ele é descentralizado e não depende de servidores específicos para funcionar. Ele possue grande quantidade de relays voluntários, que vão continuar existindo independente do número de usuários utilizando este protocolo, e é bastante resistente a maus atores. Sua desvantagem é a quantidade de passos necessários para estabelecer uma conexão, o que gera uma latência maior. O Tor também soluciona um problema muito grande da arquitetura peer-to-peer, que é a dificuldade de encontrar outros usuários. Assim, vamos propor para esse projeto um programa de arquiteura hibrida, que utiliza o Tor para encontrar outros usuários, e uma arquitetura peer-to-peer para a comunicação entre eles.

\subsection{Arquitetura híbrida}

Na arquitetura proposta por esse projeto, quando um usuário estiver online, ele deve hospedar um serviço oculto com seu par de chaves publico-privada. Quando ele quiser enviar mensagens a outro usuário, ele primeiro deve conectar-se ao serviço oculto correspondente ao nome de usuário do destinatário. Ao se conectarem, esses clientes vão trocar endereços de IP e tentar realizar uma conexão p2p. Com essa conexão eles vão realizar troca de mensagens.